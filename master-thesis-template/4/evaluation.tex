
\section{xDR Challenge 2023 環境での評価}


xDR Challenge 2023の環境についてあらためて詳細に説明する.
対象施設は高速道路のサービスエリアである.
対象の建物は2つあり,1つは2階建て,もう1つは平屋である.
提供される訓練データとスコアリングデータの詳細を表\ref{table:data}に示す.
BLEビーコンとしてMyBeacon(Aplix)\cite{beacon-aplix}を利用し,BLE信号は0.1秒ごとに発信される.
ビーコンの位置は,フロアマップの座標の (x, y, z) 位置として提供される.
競技に使用される9軸IMUセンサデータはAQUOS Sense 6(SHARP)である.
正解のデータはハンドヘルド LiDAR (GeoSLAM ZEB-Horizon) を使用して約 100 Hz で測定されたものである.
サンプリングデータは歩行時間が54秒から667秒までの全323個のデータが提供された.
スコアリングデータは歩行時間が134秒から234秒までの全9個のデータが提供された.
スコアリングはBLEビーコンのデータがないものとあるものでそれぞれ行われる.
9データのうち3個がBLEビーコンのデータが与えられず,残りの6個にはBLEビーコンのデータが与えられる.


\begin{table*}[ht]
    \centering
    \caption{提供データの概要}
    \begin{tabularx}{\textwidth}{|X|l|X|l|l|}
        \hline
        データタイプ & 測定デバイス & レート & 訓練データ & スコアリングデータ \\ \hline
        加速度 & AQUOS Sense 6 & 約100Hz & 使用可能 & 使用可能 \\ \hline
        角速度 & AQUOS Sense 6 & 約100Hz & 使用可能 & 使用可能 \\ \hline
        磁気 & AQUOS Sense 6 & 約100Hz & 使用可能 & 使用可能 \\ \hline
        BLE RSSI & AQUOS Sense 6 & 10Hzのビーコンから送信,AQUOS Sense 6で受信時に記録 & 使用可能 & 使用可能 \\ \hline
        正解位置 (Ground Truth) (x, y, z) & ZEB-Horizon & 約100Hz & 使用可能 & 始めと終わりのみ使用可能 \\ \hline
        正解姿勢 (Ground Truth) (四元数) & ZEB-Horizon & 約100Hz & 使用可能 & 始めと終わりのみ使用可能 \\ \hline
        正解階層名 (Ground Truth) & - & 各パスの1階層名 & 使用可能 & 使用可能 \\ \hline
    \end{tabularx}
    \label{table:data}
\end{table*}



xDR Challenge 2023ではPDRベンチマーク標準化委員会によって
提供された評価フレームワークを使用して評価が行われた.
このフレームワークではl\_ce(CE:円形誤差),l\_ca(CA\_l:局所空間における円形精度),
l\_eag(誤差蓄積勾配),l\_ve(VE:速度誤差),l\_obstacle(障害物回避要件)
の5つの評価指標が用いられた.
総合評価指標は式\ref{eq:evaluation_index}に示す式で計算される.
このライブラリを使って得られた評価と各指標の重みを表\ref{table:evaluation_index}に示す.
l\_ce,l\_eag, l\_ve,l\_obstacleでは一定の精度を得られた.
しかし評価項目l\_caでは精度が低い結果となった.
l\_caの値の低さは場合局所空間における位置推定誤差の分布が広さを示している.
これは環境条件の変化やセンサデータの微妙な違いが,位置推定結果に大きな影響を与える可能性が高いのを示している.
この問題を解決するためには,PDRアルゴリズムを改善し,より精度の高い位置推定を行う必要がある.

\begin{equation}
	\begin{aligned}
		I_i = & W_{ce} \times I_{ce} + W_{ca} \times I_{ca}                                        \\
		      & + W_{eag} \times I_{eag} + W_{ve} \times I_{ve} + W_{obstacle} \times I_{obstacle}
	\end{aligned}
	\label{eq:evaluation_index}
\end{equation}

\begin{table}[ht]
	\caption{評価指数の概要}
	\centering
	\begin{tabular}{l|l|l}
		\hline
		指標                        & 値 (\%) & 重み   \\ \hline
		l\_ce(CE:円形誤差)            & 88.55  & 0.25 \\
		l\_ca(CA\_l:局所空間における円形精度) & 62.51  & 0.20 \\
		l\_eag(EAG:誤差蓄積勾配)        & 93.02  & 0.25 \\
		l\_ve(VE:速度誤差)            & 95.55  & 0.15 \\
		l\_obstacle(障害物回避要件)      & 93.48  & 0.15 \\
		l (総合評価指数)                & 86.25  &      \\ \hline
	\end{tabular}
	\label{table:evaluation_index}
\end{table}



