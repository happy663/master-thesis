\chapter{はじめに}
\thispagestyle{myheadings}


\section{背景}
屋内位置推定技術は現代社会において重要な役割を果たしており様々な活用がされている.
屋内位置推定技術が使用される一例として,ショッピングモール施設内でのナビゲーションシステムが挙げられる\cite{burasapo}.
このシステムでは顧客の位置情報を元にして,現在地周辺にある店舗のおすすめ情報や目的地までのナビゲーションを提供する.
屋外における位置推定技術としてGPSが広く利用されているが,
屋内環境では建物の壁や天井がGPS衛星からの電波を遮断してしまい,
位置推定精度が大きく低下する問題がある.
そのため別のアプローチが必要とされている.

屋内位置推定の手法には絶対位置推定手法,相対位置推定手法がある.
絶対位置推定手法は特定の基準点を元に位置を推定する手法である.
その代表例としてはWi-Fi,BLEビーコン,地磁気などの情報を利用したものがある.
Wi-FiやBLEビーコンなどの電波情報を利用した手法の1つに三点測位手法がある.
この手法では3つのアクセスポイント(以下,AP)の位置情報とそこからの電波強度を基に推定を行う.

相対位置推定手法はある特定の基準点からの相対的な位置を推定する手法である.
その代表例としてPDR(Pedestian Dead Reckoning)がある.
PDRは歩行者が身につけたスマートフォンなどに搭載される加速度計,
ジャイロスコープなどのセンサを利用して歩行者の歩幅,進行方向,ステップタイミングを推定する.
その情報を元に歩行者の移動を累積的に計算し,基準位置からの相対的な位置を推定する手法である.

相対位置推定,絶対位置推定にはそれぞれ問題点がある.
絶対位置推定は特定の環境に依存しており,その環境がない場所では位置推定ができない問題点がある.
PDRによる相対位置推定には初期位置と初期進行方向の情報が必ず必要な問題がある.
またPDRではセンサーのわずかな誤差が累積し続けるため,長時間に及ぶ歩行では軌跡の形状が大きく変化してしまう問題もある.

ハイブリッド位置推定手法は複数の異なる技術やデータソースを組み合わせて,
屋内環境での位置を高精度に推定するための手法である.
その中でもPDRと環境情報を組み合わせた手法がある.
例としてPDRとWi-Fiの電波強度を組み合わせた手法ではWi-Fi APの信号強度(RSSI)を利用して位置を推定し,PDRの位置情報を補正する.
電波強度が弱い場所ではPDRの位置情報を信用し,電波強度が強い場所ではWi-Fiの位置情報を信用する.
これによって累積誤差をキャンセルし高精度の位置推定が行える.
PDRと環境情報を組み合わせたハイブリッド手法は屋内位置推定の手法の中で
有用な手法として様々な情報を用いた研究がされている.

しかしこれらの研究は特定の条件下でのPDRと組み合わせたものが多く,他の条件下では推定が難しい.
例としてWi-Fiを利用した方法の場合,基地局の位置が事前に把握できるケースとできないケースがある.
また追加で補正に利用できる情報がある場合も考えられる.
環境によって補正に使用できる情報は異なる.
多くの環境を想定した屋内位置推定手法は少ない.

\section{目的とアプローチ}

本研究では様々な環境と状況に対応できるPDRベースの屋内位置推定ライブラリの基礎検討を行う.
本研究の概要を図\ref{fig:overview}に示す.
推定に使用できる情報をセンサ情報と環境情報に分け,これらの情報を元にPDRとその補正を行うライブラリを構築する.
これらの関数はそれぞれの環境や条件に応じて適用,組み合わせて使える形を目指す.

\begin{figure}[h]
	\centering
	\includegraphics[width=120mm]{image/first.png}
	\caption{様々な状況と環境に対応できる\\PDRベースの
		屋内位置推定ライブラリの概要}    \label{fig:overview}
\end{figure}

\section{論文構成}
本稿の構成は以下の通りである.2 章では,屋内位置推定に関する研究及びPDRライブラリに
関連する事例を提示し,本研究との関連性を述べる.
3 章では,ライブラリの要求仕様とライブラリの実装について述べる.
4 章では,ライブラリの検証と他環境における適用可能性の検討について述べる.
最後に5 章では,まとめと今後の課題について述べる.


