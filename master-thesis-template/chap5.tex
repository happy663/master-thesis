\chapter{おわりに}
\thispagestyle{myheadings}


\section{まとめ}
本論文では,環境情報などを利用したPDRベースの位置推定ライブラリの基礎検討を行った.
PDRはスマートフォンなどの機器さえあれば環境に左右されず一定の推定が可能である.
一方でPDRは相対的な手法であるため初期位置,初期進行方向が不明な問題や
時間の経過に応じて特有の誤差が蓄積する問題がある.
そのため環境情報などを使用して補正するハイブリット手法が用いられる場合が多い.
しかしハイブリット手法は特定の環境を想定したものが多く,複数の環境を想定したものは多くない.
そこで本研究では様々な状況と環境に対応できるPDRベースの屋内位置推定ライブラリの基礎検討を行った.
補正に利用できる情報をセンサ情報,環境情報,その他の3つに分類し,それぞれの情報を用いた補正処理を提案した.
その結果として,xDR Challenge 2023環境下では一定の精度を獲得した.
また他環境においても本ライブラリが適用可能であるか検討を行った.


\section{今後の課題}
課題としてはPDRアルゴリズムの改善が挙げられる.
歩幅や歩行タイミングの精度の向上によって位置推定の精度向上が期待できる.
また本論文は2次元の屋内位置推定のみを想定したライブラリ構成となっている.
現実の屋内では3次元で構成されるものが多いため,本ライブラリを3次元空間に適用できるような拡張を検討したい.
具体的にはスマートフォンの気圧センサを使用すれば相対的な階層間の移動の検知が可能である.
これとフロアマップ情報を組み合わせによって3次元空間での位置推定が実現できると考えられる.

